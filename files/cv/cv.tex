\documentclass[dvipdfmx]{jsarticle}
%
\usepackage{amsmath}
\usepackage{amssymb}
\usepackage{amsfonts}
\usepackage[dvipdfmx]{graphicx}
\usepackage[dvipdfmx]{hyperref}%ハイパーリンクの埋め込み
%\usepackage{mediabb}%pdfを簡単に取り込み
\usepackage{here}
\usepackage{yhmath}%長いtildeのため
\usepackage[dvipdfmx]{color}
\usepackage{ulem}%取り消し線のため
\usepackage{bm}
\allowdisplaybreaks[4]%式変形中の改ページの許可
%\usepackage[dvipdfmx]{hyperref}%ハイパーリンクの埋め込み
%\usepackage{pxjahyper} %%hyperref読み込みの直後に読み込んでおくおまじない
\usepackage{comment}
\usepackage{braket}
%
%おまじない
\setlength{\textwidth}{\fullwidth}
\setlength{\textheight}{39\baselineskip}
\addtolength{\textheight}{\topskip}
\setlength{\voffset}{-0.5in}
\setlength{\headsep}{0.3in}
\setlength{\abovedisplayskip}{3pt}%上部のマージン
\setlength{\belowdisplayskip}{3pt}%下部のマージン
%
\pagestyle{empty}%ページ数を非表示にする
%
%
%
%
\begin{document}
%\maketitle
%
%
%
%
\begin{flushright}
Mar. 17, 2021
\end{flushright}
%
%
%
%
{\bf\Large Akihito Yoneyama (米山 瑛仁)}
\vspace{3mm}
\par
Institute of Physics, Graduate School of Arts and Sciences, the University of Tokyo
\par
Email: yoneyama.aki@gmail.com/yoneyama@gokutan.c.u-tokyo.ac.jp
%
%
%
%
{\ }\\\\
\vspace{3mm}
{\bf\large Education}
\vspace{-6mm}
\\\hrulefill
\begin{itemize}
\item
Apr. 2020 - 
\par
Doctoral Course
\par
Institute of Physics, Graduate School of Arts and Sciences, the University of Tokyo
\par
Supervisor: Prof. Atsuo Kuniba
\item
Apr. 2018 - Mar. 2020
\par
Master Course
\par
Institute of Physics, Graduate School of Arts and Sciences, the University of Tokyo
\par
Supervisor: Prof. Atsuo Kuniba
\item
Apr. 2014 - Mar. 2018
\par
Department of Physics, the University of Tokyo
\end{itemize}
%
%
%
%
{\ }\\
\vspace{3mm}
{\bf\large Work Experience}
\vspace{-6mm}
\\\hrulefill
\begin{itemize}
\item
Oct. 2020 - 
\par
Research Internship (part-time)
\par
GCI Asset Management Kyoto Lab, Kyoto, Japan
%
%
\item
Aug. 2019 - Sep. 2019
\par
Research Internship (full-time)
\par
Preferred Networks, Inc., Tokyo, Japan
\end{itemize}
%
%
%
%
{\ }\\
\vspace{3mm}
{\bf\large Academic Work Experience}
\vspace{-6mm}
\\\hrulefill
\begin{itemize}
\item
Sep. 2020 - Jan. 2021
\par
Teaching assistant for the course \textit{Electromagnetics B}
\par
The University of Tokyo, Tokyo, Japan
\end{itemize}
%
%
%
%
{\ }\\
\vspace{3mm}
{\bf\large Award}
\vspace{-6mm}
\\\hrulefill
\begin{itemize}
\item
Mar. 2020
\par
Encouragement Award, Graduate School of Arts and Sciences, the University of Tokyo
% (広域科学専攻奨励賞)
\end{itemize}
%
%
%
%
{\ }\\
\vspace{3mm}
{\bf\large Paper}
\vspace{-6mm}
\\\hrulefill
\begin{enumerate}
% 論文を逆順で表示するためのカウンター
\newcounter{paperCounter}
\renewcommand*\theenumi{\the\numexpr\value{paperCounter}-\value{enumi}}
\setcounter{paperCounter}{5}
\item
A.Yoneyama, ``Boundary from bulk integrability in three dimensions: 3D reflection maps from tetrahedron maps'', \href{https://arxiv.org/abs/2103.01105}{arXiv:2103.01105}
%
%
\item
A.Yoneyama, ``Tetrahedron and 3D reflection equation from PBW bases of the nilpotent subalgebra of quantum superalgebras'', \href{https://arxiv.org/abs/2012.13385}{arXiv:2012.13385}
%
%
\item
A.Kuniba, M.Okado and A.Yoneyama, ``Reflection $K$ matrices associated with an Onsager coideal of $U_p(A_{n-1}^{(1)}),U_p(B_n^{(1)}),U_p(D_n^{(1)})$ and $U_p(D_{n+1}^{(2)})$'', \href{https://iopscience.iop.org/article/10.1088/1751-8121/ab3715}{J. Phys. A: Math. Theor. {\bf 52} 375202 27pages (2019)}, \href{https://arxiv.org/abs/1904.05653}{arXiv:1904.05653}
%
%
\item
A.Kuniba, M.Okado and A.Yoneyama, ``Matrix product solution to the reflection equation associated with a coideal subalgebra of $U_q(A_{n-1}^{(1)})$'', \href{http://links.springernature.com/f/a/VdNxbTDs8BDjk4rQBBmAjw~~/AABE5gA~/RgRekPU6P0QwaHR0cDovL3d3dy5zcHJpbmdlci5jb20vLS8wL0FXb09qTHhwT3VHTGE4WnNvSjRvVwNzcGNCCgAAusGvXDBnf-5SIHlvbmV5YW1hQGdva3V0YW4uYy51LXRva3lvLmFjLmpwWAQAAAbn}{Lett. Math. Phys. {\bf 109} 2049-2067 (2019)}, \href{https://arxiv.org/abs/1812.03767}{arXiv:1812.03767}
\end{enumerate}
%
%
%
%
{\ }\\
\vspace{3mm}
{\bf\large Oral Presentation at International Conference}
\vspace{-6mm}
\\\hrulefill
\begin{enumerate}
% 論文を逆順で表示するためのカウンター
\newcounter{paperCounter2}
\renewcommand*\theenumi{\the\numexpr\value{paperCounter2}-\value{enumi}}
\setcounter{paperCounter2}{2}
\item
Mar. 5-7, 2019 @ the University of Tokyo (Invited)
\par
``Matrix product solution to the reflection equation associated with a coideal subalgebra of $U_q(A_{n-1}^{(1)})$'', \href{https://sites.google.com/view/ia19/home}{Infinite Analysis 19 Quantum Symmetries and Integrable Systems}
\end{enumerate}
%
%
%
%
{\ }\\
\vspace{3mm}
{\bf\large Invited Seminar}
\vspace{-6mm}
\\\hrulefill
\begin{enumerate}
% 論文を逆順で表示するためのカウンター
\newcounter{paperCounter3}
\renewcommand*\theenumi{\the\numexpr\value{paperCounter3}-\value{enumi}}
\setcounter{paperCounter3}{3}
\item
Jan. 14, 2021 @ the University of Tokyo (Online) (Host: Ralph Willox)
\par
``Tetrahedron and 3D reflection equation from PBW bases of the nilpotent subalgebra of quantum superalgebras'', \href{https://www.ms.u-tokyo.ac.jp/seminar/discrete_e/past_e.html}{Discrete Mathematical Modelling Seminar}
%
%
\item
Apr. 10, 2019 @ Rikkyo University (Host: Jimbo Michio)
\par
``Review about tetrahedron equation and technical details about [KOY18]''
\end{enumerate}
%
%
%
%
{\ }\\
\vspace{3mm}
{\bf\large Oral Presentation at Domestic Conference}
\vspace{-6mm}
\\\hrulefill
\begin{enumerate}
% 論文を逆順で表示するためのカウンター
\newcounter{paperCounter4}
\renewcommand*\theenumi{\the\numexpr\value{paperCounter4}-\value{enumi}}
\setcounter{paperCounter4}{3}
\item
Mar. 15-18, 2021 @ Keio University (Online)
\par
``Tetrahedron and 3D reflection equation from PBW basis of the nilpotent subalgebra of quantum superalgebras'', \href{https://old.mathsoc.jp/en/meeting/keio21mar/}{Mathematical Society of Japan Spring Meeting 2021}
%
%
\item
Feb. 10-14, 2021 @ Online
\par
``Tetrahedron equation from PBW bases of the nilpotent subalgebra of quantum superalgebras'', \href{https://sites.google.com/view/math-graduate/MATHSCI-FRESHMAN-SEMINAR/2021}{Mathsci Freshman Seminar 2021}
\end{enumerate}
%
%
%
%
\newpage
{\ }\\
\vspace{3mm}
{\bf\large Skill}
\vspace{-6mm}
\\\hrulefill
\par
Mathematica, Python, C/C++
%
%
%
%
\end{document}
